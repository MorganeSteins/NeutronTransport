\documentclass[11pt,a4paper]{article}
\usepackage[utf8]{inputenc}
\usepackage[top=2.5cm, bottom=2.8cm, left=2.7cm, right=2.7cm]{geometry} 
\usepackage{xcolor}
\usepackage{amsmath}
\usepackage{amsfonts,stmaryrd}
\usepackage{amsthm}
\usepackage{amssymb}
\usepackage{natbib}
\usepackage{listings}
\usepackage{graphicx}
\usepackage{caption}
\usepackage{subcaption}

\newcommand{\pp}{\varphi}
\newcommand{\st}{\Sigma_t}
\newcommand{\into}{\int_{-1}^{1}}

\title{TP1 AMS302 - Modélisation et Simulation du transport de particules neutres}
\author{Morgane STEINS et Etienne PEILLON}
\date{Version préliminaire pour le 14/10/2019}

\begin{document}
\maketitle
\vspace{0.5cm}

Nous cherchons à résoudre l'équation du transport neutronique en 1D sur l'intervalle $[0;1]$ et pour tout $\mu \in [1;1]$
\begin{equation}
    \label{eq.transport}
    \mu \frac{\partial \pp}{\partial x} (x,\mu) + \st \pp(x,\mu) = \frac{1}{2} \into \pp(x,\mu')\ d\mu' + S(x,\mu)
\end{equation}



\section{Solveur Monte-Carlo}

\subsection{Préliminaires}
Tout d'abord nous justifions la forme du libre parcours d'un neutron. Une surface efficace macroscopique $\Sigma$ caractérise la probabilité d'intéraction sur un petit intervale $\delta l$ 
$$P_{\text{interaction}}([x;x+\delta l]) = \Sigma \delta l$$ 
Lorsque l'on projette cette distance sur l'axe des abscisses, on fait apparaitre le cosinus de l'angle entre le vecteur directeur du déplacement et $\vec{x}$ qui est définit par $\mu$ dans ce TP.
D'où en 1D $P_{\text{interaction}}(x) = \frac{\Sigma}{\mu} \delta l $

On peut voir le libre parcours du neutron de deux manières différentes. 
La première est de considérer qu'il s'agit d'une sorte comme un "temps" de survie et utiliser le fait que ce genre de phénomène se modélise avec une loi exponentielle. 
On peut aussi utiliser faire une démonstration plus physique. 
On découpe l'intervalle $[0;1]$ en $N$ sous-intervalles de longeur $dx$ et on considère partir du point 0. 
La probabilité d'interaction au point $x=idx$ s'obtient en considérant qu'il y a une intercation dans l'intervalle $i$ mais pas dans les précédents ce qui donne
\begin{equation*}
    P_{\text{interaction}}(x \in [idx;(i+1)dx]) = \frac{\Sigma}{\mu}(1-\frac{\Sigma}{\mu}dx)^{i}
\end{equation*}

Le passage à la limite $N \rightarrow \infty$ donne la densité de probabilité d'interagir au point $x$ et pas avant, donc la densité de probabilité du libre parcours $f$:
$f(x) = \frac{\Sigma}{\mu}e^{-\frac{\Sigma}{\mu}x}$

On vérifie immédiatement que $\int_0^{\infty} f(x) \text{d}x = 1$.

Pour échantillonner une variable aléatoire selon une loi, on peut inverser sa fonction de répartition si elle est connue. 
La fonction de répartition $F$ du libre parcours moyen s'écrit
\begin{equation*}
    F(x) = \int_0^x f(t) \text{d}t = \int_0^x \frac{\Sigma}{\mu}e^{-\frac{\Sigma}{\mu}t} = 1- e^{-\frac{\Sigma}{\mu}x}
\end{equation*}

Pour l'inverser on résoud $y = F(x)$
\begin{equation*}
    y = 1- e^{-\frac{\Sigma}{\mu}x}  \iff  x = \frac{\mu}{\Sigma} \text{log}(1-y) = F^{-1}(y)
\end{equation*}

Pour tirer une variable aléatoire selon la loi $F$, on tire $y$ selon une loi uniforme sur $[0;1]$ et on calcule $F^{-1}(y)$.

\vspace{0.1cm}
La méthode Monte-Carlo consiste à simuler un grand nombre de trajectoires de neutrons et en déduire le flux neutronique par la loi des grands nombres. 


\newpage
\begin{itemize}
    \item faire l'explication de l'estimateur du flux neutronique
    \item Cas homogène ponctuel
    \item Cas homogène uniforme
    \item Cas non homogène
    \item Cas scattering : le plus dur il faut coder les $n$ rebonds des particules
    \item Etudier les temps de calcul
    \item Etudier la convergence ($\sqrt{N}$ théoriquement)
    \item Etudier la complexité ?? 
    \item blabla 
\end{itemize}

\end{document}